\documentclass[12pt]{article}

\usepackage{styles/log-style}

\begin{document}
\begin{titlepage}
\begin{center}
\bfseries

{\Large Министерство науки и высшего образования}

\vspace{12pt}

{\Large Московский Авиационный Институт \\ (национальный исследовательский университет)}

\vspace{48pt}

\large Институт информационных технологий и прикладной математики

\vspace{36pt}

\large Кафедра вычислительной математики и программирования

\vspace{72pt}

Журнал по ознакомительной практике \\
(индивидуальный план)

\end{center}

\vspace{180pt}

\begin{flushleft}
Студент: И. И. Иванов \\
Группа: М8О-119Б-29 \\
Оценка: \\
Дата: \\
Подпись:
\end{flushleft}

\vspace*{\fill}

\begin{center}
\bfseries
Москва, \the\year
\end{center}
\end{titlepage}

\pagebreak

\input{formal/instruction}
\begin{center}
\bfseries{\large ЗАДАНИЕ}
\end{center}

Принять участие в учебно-тренировочных контестах по олимпиадному программированию для студентов первого курса в течении 9 дней: посетить и проработать установочные лекции, решать и дорешивать конкурсные задания, принять участие в разборах контестов. Составить отчёт в форме журнала установленной формы и пройти процедуру защиты практики.

Объём практики 108 часов в течение 12 учебных дней.

\vspace*{\fill}
Руководитель практики от института:

\vspace{5pt}
\enquote{29} $\underset{\text{(дата)}}{\uline{\text{июня}}}$ 2022\,г.\hfill \tline{(подпись)}{1in}
\pagebreak

\begin{center}
\bfseries{\large ТАБЕЛЬ ПРОХОЖДЕНИЯ ПРАКТИКИ}
\end{center}

\resizebox{\columnwidth}{!}{
\begin{tabular}{|c|c|c|c|c|c|c|c|}
\hline
\textbf{№} & \textbf{Дата} & \textbf{Название} & \textbf{Время} & \textbf{Место} & \textbf{Решено} & \textbf{Дорешано} & \textbf{Подпись} \\
& & \textbf{контеста} & \textbf{проведения} & \textbf{проведения} & \textbf{задач} & \textbf{задач} & \\
\hline
1 & 17.09.2021 & Выдача задания, Основы C++ [1] & 16:30 - 21:30 & МАИ & 1 & 2 & \\
\hline
2 & 01.10.2021 & Библиотека C++ [3] & 16:30 - 21:30 & Дистанционно & 1 & 2 & \\
\hline
3 & 21.11.2021 & RuCode 4.0 Div A-B Champoinship & 11:00 - 16:00 & Дистанционно & 1 & 2 & \\
\hline
4 & 05.12.2021 & Осенняя олимпиада первого курса & 11:00 - 17:00 & МАИ & 1 & 2 & \\
\hline
5 & 27.01.2022 & Codeforces Round \#768 (Div. 2) & 17:35 - 19:35 & Дистанционно & 1 & 2 & \\
\hline
6 & 11.03.2022 & Паросочетания в двудольном графе, & 16:30 - 21:30 & Дистанционно & 1 & 2 & \\
& & потоки в транспортной сети [15] & & & & & \\
\hline
7 & 12.07.2022 & Оформление журнала. & 9:00 - 18:00 & МАИ & & & \\
& & Защита практики & & & & & \\
\hline
& & Итого часов & 108 & & & & \\
\hline
\end{tabular}
}

\pagebreak

\begin{center}
\bfseries{\large Отзывы цеховых руководителей практики}
\end{center}

Принято участие в $26$ контестах, решено $99$ и дорешано $26$ задач контестов, оформлен журнал практики с электронным приложением. Задание практики выполнено.

Рекомендую оценку

\vspace{15pt}

\hfill Тренер Зайцев В. Е. \tline{(подпись)}{1in}

\vspace{200pt}

\begin{center}
\bfseries{\large Работа в помощь предприятию}
\end{center}

Встречи с представителями ИТ-компаний, сотрудничающих с МАИ.

\pagebreak

\begin{center}
\bfseries{\large ПРОТОКОЛ }

\vspace{12pt}

\bfseries{ЗАЩИТЫ ТЕХНИЧЕСКОГО ОТЧЁТА}
\end{center}
\noindent
по {\itshape ознакомительной практике}

\vspace{8pt}
\noindent
студентом:
\noindent
Ивановым Иваном Ивановичем

\begin{longtable}{p{7cm}|p{11cm}}
    \hline
    {\bfseries Слушали:} & {\bfseries Постановили:}  \\
    % Отчёт практиканта & Считать практику выполненной \\
    % & и защищённой на \\
    \rule{0pt}{450pt} & Общая оценка: \underline{\hspace{2in}}\\
    \rule{0pt}{15pt} & \\
    \hline
\end{longtable}

\vfill

\noindent\begin{tabular}{@{}l l l}
Председатель: & Зайцев В. Е. & \underline{\hspace{2in}} \\
Члены: & Сорокин С. А. & \underline{\hspace{2in}} TODO или кто??? \\
& Инютин М. А. & \underline{\hspace{2in}} TODO или кто???
\end{tabular}
\vspace{12pt}

\noindent
Дата: 12 июля \the\year\,г.
\pagebreak


\pagebreak

% \includepdf[pages=10, scale=0.75, pagecommand=\subsection*{Grand Prix of Dolgoprudny 12.09.2021}]{statements/210830.pdf}
% \subsubsection*{Идея}
% Тут вы описываете идею решения, оцениваете сложность...
%
% Например, сложность жадного алгоритма $O(n \cdot \log{n})$, а перебор $O(2 ^ {n} \cdot n ^ 2)$.
% \subsubsection*{Исходный код}
% \lstinputlisting{src/gp_dolgop_g.cpp}
% \subsubsection*{Положение команды}
% \includegraphics[scale=0.25]{images/gp_dolgop.png}\newline\noindent

\end{document}
